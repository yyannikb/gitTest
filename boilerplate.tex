\usepackage[english]{babel}
\usepackage{listings}
\usepackage{color}
\definecolor{Gray}{gray}{0.95}

% Auto-title
\makeatletter
\DeclareRobustCommand{\getlstname}{%
    \begingroup
    % \lstname seems to change hyphens into \textendash
    \def\textendash{-}%
    \filename@parse{\lstname}%
    \filename@base.\filename@ext%
    \endgroup
}
\makeatother

% fix the high tilde
\usepackage{url}
\lstset{literate=%
    {~}{{\url{~}}}1
}

\lstdefinestyle{code}{
    IN TESTBR, THIS SHOULD LEAD TO A CONFLIG!
    In master: adding one more line to \lstdefinestyle
    basicstyle = \small\ttfamily, % Font of the text
    numbers = left,               % Position of the line numbers
    numberstyle = \footnotesize,  % Style of line numbers
    frame = tb,                   % Style of surrounding frame
    framextopmargin=.75mm,        % Space margin top
    framexbottommargin=.75mm,     % Space margin bottom
    framexleftmargin=2mm,         % Space margin left
    framexrightmargin=2mm,        % Space margin right
    tabsize = 4,                  % Size of tab character
    breaklines = false,           % Wrap lines of code that are too long
    *************************************
    HERE I ADD ANOTHER LINE IN TEST BRANCH
    *************************************
    columns = fullflexible,     
    showstringspaces = false,
    backgroundcolor = \color{Gray},
    title = \getlstname
}
\lstdefinestyle{text}{
    basicstyle = \small\ttfamily, % Font of the text
    frame = tb,                   % Style of the surrounding frame
    framextopmargin=.75mm,        % Space margin top
    framexbottommargin=.75mm,     % Space margin bottom
    framexleftmargin=2mm,         % Space margin left
    framexrightmargin=2mm,        % Space margin right
    tabsize = 3,                  % Size of tab character
    breaklines = true,            % Wrap lines of text that are too long
    columns = flexible,     
    showstringspaces = false,
    backgroundcolor = \color{Gray},
    title = \getlstname
}
\lstdefinestyle{inline}{
    basicstyle = \small\ttfamily, % Font of the text
    frame = tb,                   % Style of surrounding frame
    framextopmargin=.75mm,        % Space margin top
    framexbottommargin=.75mm,     % Space margin bottom
    framexleftmargin=2mm,         % Space margin left
    framexrightmargin=2mm,        % Space margin right
    tabsize = 4,                  % Size of tab character
    breaklines = false,           % Wrap lines of code that are too long
    columns = fullflexible,     
    showstringspaces = false,
    backgroundcolor = \color{Gray}
}
\lstset{keepspaces=true}

\renewcommand{\tt}[1]{\texttt{{#1}}}
